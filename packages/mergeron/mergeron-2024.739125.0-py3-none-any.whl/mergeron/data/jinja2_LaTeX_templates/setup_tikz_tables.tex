% Tables in tikz, but first we need to define some colors for nonwhite text backgrounds
\definecolor{OBSHDRFill}{HTML}{0a6c97}
\definecolor{SIMHDRFill}{HTML}{646464}
\definecolor{DataFill}{HTML}{dfeadf}
% The below are definition's from Paul Tol's website, https://personal.sron.nl/~pault/
\definecolor{VibrBlue}{HTML}{0077bb}
\definecolor{BrightGreen}{HTML}{228833}
\definecolor{HiCoYellow}{HTML}{ddaa33}
\definecolor{VibrRed}{HTML}{cc3311}
% Some useful documents from grokking various options used below
% https://courses.helsinki.fi/sites/default/files/course-material/4611024/LaTeXandFriendsCourseInHelsinki_9_v2.pdf
% https://tug.org/TUGboat/tb39-1/tb121duck-tikz.pdf
% https://tug.org/TUGboat/tb41-1/tb127duck-matrix.pdf
% https://tex.stackexchange.com/questions/22286/coloring-every-other-row-of-a-table-with-vertical-lines
% https://tex.stackexchange.com/questions/354043/swot-matrix-with-multicolumn-and-multirow
% https://tex.stackexchange.com/questions/20599/horizontal-row-separation-line-in-tikz-matrix-like-hline-in-tabular
% https://tex.stackexchange.com/questions/63560/how-to-use-x-coordinate-of-a-point-and-y-coordinate-of-other-point
% https://tex.stackexchange.com/questions/167944/auto-word-wrapping-in-a-node-in-tikz
% https://tex.stackexchange.com/questions/132741/typewriter-in-tikz-node
\tikzset{
    green_bar_filler/.code={%
        \pgfmathtruncatemacro{\itest}{ifthenelse(%
            \pgfmatrixcurrentcolumn==1,0,%
            ifthenelse(iseven(\pgfmatrixcurrentrow),1,3))}
        %\typeout{\the\pgfmatrixcurrentcolumn,\pgfmatrixcurrentrow->\itest}
        \ifcase\itest
        \tikzset{fill=none}%
        \or
        \tikzset{fill=DataFill}%
        \else
        \tikzset{fill=white}%
        \fi
    },
    get matrix dims/.code={
        \global\tnc=\pgf@matrix@numberofcolumns
        \global\tnr=\pgfmatrixcurrentrow
    },
    % If you only have numbers, text depth = 0ex; if text, text depth = 0.25ex, (may need tweaking for alignment across cells)
    anytext/.style = {
        font = \sffamily\scriptsize\addfontfeatures{Numbers={Monospaced,Lining}},
        align = center,
        inner sep = 3pt,
        text depth = 0pt,
    },
    hdrtext/.style = {
        anytext,
        text = white,
        fill = OBSHDRFill,
    },
    notetext/.style = {
        anytext,
        font = \sffamily\tiny,
        align = left,
    },
    anytable/.style = {
        matrix of nodes,
        nodes in empty cells,
        column sep = -1.0\pgflinewidth,
        row sep = -1.0\pgflinewidth,
        inner sep = -0.25\pgflinewidth,
        outer sep = -0.25\pgflinewidth,
        nodes = {anchor = center, minimum height = 12.5pt,}
    },
    datatable/.style = {
        anytable,
        nodes = {
            anytext,
            align = right,
            draw = none,
            fill = none,
            green_bar_filler,
        },
    },
    hdrtable/.style = {
        anytable,
        nodes = {
            hdrtext,
            draw = none,
            fill = OBSHDRFill,
        },
    },
    hrow/.style = {
        hdrtable,
        nodes = {text width = 40pt,},
    },
    hcol/.style = {
        hdrtable,
        nodes = {text width = 60pt,},
    },
}
% Define layers for later reference
% https://tex.stackexchange.com/questions/40840/put-a-node-behind-another-in-a-tikz-diagram
\pgfdeclarelayer{background}
\pgfsetlayers{background,main}


